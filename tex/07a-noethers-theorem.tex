\documentclass{article}

\usepackage{amsmath,amssymb}
\usepackage{mathrsfs}
\usepackage{braket}
\usepackage{tensor}

\usepackage{textpos}
\usepackage{eso-pic}
\usepackage{tikz}
\usetikzlibrary{tikzmark}

\title{Noether's Theorem}
\author{Anthony Mezzacappa}
\date{September 15 and 17, 2020}

\begin{document}

\setlength{\parskip}{1em}

\maketitle

\noindent If $\phi$ satisfies the Euler-Lagrange EOM ($\delta S = 0$), for \underline{any} variation we have
\begin{equation*}
    \delta \mathscr{L} = \partial_\mu \left( \dfrac{ \partial \mathscr{L} }{ \partial ( \partial_\mu \phi ) } \delta \phi \right) \tikzmarknode{deltaL}{}
\end{equation*}

{%
\begin{textblock*}{3.25in}(2.75in,-0.5in)%
\begin{minipage}[h!]{3.25in}
    \begin{align*}
        \tikzmarknode{inset}{~} \delta \mathscr{L} &= \bigg[ \underbrace{ \dfrac{ \partial \mathscr{L} }{ \partial \phi } - \partial_\mu \left( \dfrac{ \partial \mathscr{L} }{ \partial ( \partial_\mu \phi ) } \right) }_{=0} \bigg] \delta \phi \\
        &\quad + \partial_\mu \left( \dfrac{ \partial \mathscr{L} }{ \partial ( \partial_\mu \phi ) } \delta \phi \right) \\
        &= \partial_\mu \left( \quad \right)
    \end{align*}
\end{minipage}%
\end{textblock*}%
}

{%
\begin{textblock*}{3.25in}(3.6in,0.75in)%
\begin{minipage}[h!]{3.25in}
    \begin{align*}
        \delta S &= \int \mathrm{d}^4 x~ \delta \mathscr{L} \\
        &= \int \mathrm{d}^4 x~ \partial_\mu ( \quad ) \\
        &= 0
    \end{align*}
\end{minipage}%
\end{textblock*}%
}

\begin{tikzpicture}[overlay, remember picture]
    \draw[overlay,->] (inset.north) -- (deltaL.west);
\end{tikzpicture}

\noindent Assume the action also has the symmetry
\begin{equation*}
    \delta_\Delta S = 0
\end{equation*}
under the \underline{specific} transformation
\begin{equation*}
    \phi \longrightarrow \phi + \Delta
\end{equation*}

\noindent Given the action is invariant under the transformation even though the \linebreak Lagrangian density may not be,
\begin{equation} \label{l8a-eq1}
    \delta_\Delta \mathscr{L} = \partial_\mu \left( \dfrac{ \partial \mathscr{L} }{ \partial ( \partial_\mu \phi ) } \Delta \right) \neq 0
\end{equation} % Equation 1
\begin{equation} \label{l8a-eq2}
    \delta_\Delta \mathscr{L} = \partial_\mu \tensor{K}{^\mu}
\end{equation} % Equation 2
\begin{equation*}
    \delta_\Delta S = \int \mathrm{d}^4 x~ \delta_\Delta \mathscr{L} = \int \mathrm{d}^4 x~ \partial_\mu \tensor{K}{^\mu} = 0
\end{equation*}
-- i.e., $\delta_\Delta \mathscr{L}$ must be a total divergence. Equating \eqref{l8a-eq1} and \eqref{l8a-eq2}, we get
\begin{equation*}
    \partial_\mu \left( \dfrac{ \partial \mathscr{L} }{ \partial ( \partial_\mu \phi ) } \Delta - \tensor{K}{^\mu} \right) = 0
\end{equation*}
-- i.e., we have a \underline{conserved current} ($\partial_\mu \tensor{j}{^\mu} = 0$)
\begin{equation*}
    \tensor{j}{^\mu} \equiv \dfrac{ \partial \mathscr{L} }{ \partial ( \partial_\mu \phi ) } \Delta - \tensor{K}{^\mu}
\end{equation*}
so, with a symmetry there is a conserved current.

% Page 1a

\noindent\rule{\textwidth}{.5pt}

\noindent A conserved current will be associated with a conserved "charge"
\begin{equation*}
    \partial_\mu \tensor{j}{^\mu} = \partial_0 \tensor{j}{^0} + \partial_i \tensor{j}{^i}
\end{equation*}

\noindent Integrate over all of space
\begin{equation*}
    \int \mathrm{d}^3 x~ \left( \partial_0 \tensor{j}{^0} + \partial_i \tensor{j}{^i} \right) = 0
\end{equation*}

\noindent Then
\begin{align*}
    \partial_0 \int \mathrm{d}^3 x~ \tensor{j}{^0} &= - \int \mathrm{d}^3 x~ \partial_i \tensor{j}{^i} \\
    &= 0
\end{align*}
and we identify
\begin{equation*}
    Q \equiv \int \mathrm{d}^3 x~ \tensor{j}{^0}
\end{equation*}
with the conserved "charge".

\noindent\rule{\textwidth}{.5pt}

% Back to Page 1

\noindent How do we find $\tensor{K}{^\mu}$?

% Page 2

\noindent Consider the \underline{translation invariance} of the action under the transformation
\begin{equation*}
    \tensor{x}{^\mu} \longrightarrow \tensor{x'}{^\mu} = \tensor{x}{^\mu} - \tensor{a}{^\mu}
\end{equation*}

% Page 2a

\noindent\rule{\textwidth}{.5pt}

\noindent\rule{\textwidth}{.5pt}

% Back to Page 2

\end{document}