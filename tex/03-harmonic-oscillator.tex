\documentclass{article}

\usepackage{amsmath,amssymb}
\usepackage{mathrsfs}
\usepackage{braket}

\title{The Harmonic Oscillator}
\author{Anthony Mezzacappa}
\date{September 1, 2020}

\begin{document}
\setlength{\parskip}{1em}
\maketitle

\vspace{-24pt}
\begin{align*}
H &= \frac{p^2}{2m} + \frac{1}{2} m \omega^2 x^2 \\
  &= \frac{1}{2m} \left[ p^2 + {\left( m \omega x \right)}^2 \right]
\end{align*}

\noindent This has the form $u^2 + v^2$, which can be written as
\begin{equation*}
u^2 + v^2 = ( i u + v ) ( -i u + v )
\end{equation*}

\noindent Classically, we are dealing with functions. Quantum mechanically we are dealing with operators, so we have to be careful about ordering.

\noindent The Hamiltonian operator is
\begin{equation*}
\hat{H} = \frac{\hat{p}^2}{2m} + \frac{1}{2} m \omega^2 \hat{x}^2
\end{equation*}

\noindent Define the operators
\begin{align*}
\hat{a} &\equiv \dfrac{1}{\sqrt{2 m \omega}} ( m \omega \hat{x} + i \hat{p} ) \\
\hat{a}^\dagger &\equiv \dfrac{1}{\sqrt{2 m \omega}} ( m \omega \hat{x} - i \hat{p} )
\end{align*}

% Page 2

\noindent Now consider the product
\begin{align*}
\hat{a}^\dagger \hat{a} &= \dfrac{1}{\sqrt{2 m \omega}} ( m \omega \hat{x} - i \hat{p} ) ( m \omega \hat{x} + i \hat{p} ) \\
&= \dfrac{1}{\sqrt{2 m \omega}} \left\lbrace m^2 \omega^2 \hat{x}^2 + i m \omega [ \hat{x}, \hat{p} ] + \hat{p}^2 \right\rbrace
\end{align*}
and the product
\begin{align*}
\hat{a} \hat{a}^\dagger &= \dfrac{1}{\sqrt{2 m \omega}} ( m \omega \hat{x} - i \hat{p} ) ( m \omega \hat{x} + i \hat{p} ) \\
&= \dfrac{1}{\sqrt{2 m \omega}} \left\lbrace m^2 \omega^2 \hat{x}^2 - i m \omega [ \hat{x}, \hat{p} ] + \hat{p}^2 \right\rbrace
\end{align*}

\noindent The sum
\begin{align*}
\hat{a}^\dagger \hat{a} + \hat{a} \hat{a}^\dagger &= m \omega^2 \hat{x}^2 + \frac{\hat{p}^2}{ m \omega } \\
&= \tfrac{2}{\omega} \hat{H}
\end{align*}

\noindent Then
\begin{equation*}
\hat{H} = \tfrac{1}{2} \omega ( \hat{a}^\dagger \hat{a} + \hat{a} \hat{a}^\dagger )
\end{equation*}

% Page 3

\noindent Let's consider the commutator of the operators $\hat{a}$ and $\hat{a}^\dagger$. ($\hbar = 1$)
\begin{align*}
[ \hat{a}, \hat{a}^\dagger ] &= \dfrac{1}{2 m \omega} \left\lbrace ( m \omega \hat{x} + i \hat{p} ) ( m \omega \hat{x} - i \hat{p} ) - ( m \omega \hat{x} - i \hat{p} ) ( m \omega \hat{x} + i \hat{p} ) \right\rbrace \\
&= \dfrac{1}{2 m \omega} \left\lbrace m^2 \omega^2 \hat{x}^2 - i m \omega [ \hat{x}, \hat{p} ] + \hat{p}^2 - ( m^2 \omega^2 \hat{x}^2 + i m \omega [ \hat{x}, \hat{p} ] + \hat{p}^2 ) \right\rbrace \\
&= -i ( i ) \nonumber\\
&= +1
\end{align*}

\noindent That is
\begin{equation*}
[ \hat{a}, \hat{a}^\dagger ] = 1 \qquad \Rightarrow \qquad \hat{a} \hat{a}^\dagger = 1 + \hat{a}^\dagger \hat{a}
\end{equation*}
and of course
\begin{equation*}
[ \hat{a}, \hat{a} ] = 0
\end{equation*}
\begin{equation*}
[ \hat{a}^\dagger, \hat{a}^\dagger ] = 0
\end{equation*}

\noindent Then
\begin{align*}
\hat{H} &= \tfrac{1}{2} \omega ( \hat{a}^\dagger \hat{a} + \hat{a} \hat{a}^\dagger ) \\
&= \tfrac{1}{2} \omega ( \hat{a}^\dagger \hat{a} + 1 + \hat{a}^\dagger \hat{a} ) \\
&= \omega ( \hat{a}^\dagger \hat{a} + \tfrac{1}{2} )
\end{align*}

% Page 4

\noindent Now look at
\begin{align*}
[ \hat{H}, \hat{a} ] &= \omega \left\lbrace ( \hat{a}^\dagger \hat{a} + \tfrac{1}{2} ) \hat{a} - \hat{a} ( \hat{a}^\dagger \hat{a} + \tfrac{1}{2} ) \right\rbrace \\
&= \omega ( \hat{a}^\dagger \hat{a} \hat{a} - \hat{a} \hat{a}^\dagger \hat{a} ) \nonumber\\
&= \omega [ \hat{a}^\dagger \hat{a} \hat{a} - ( 1 + \hat{a}^\dagger \hat{a} ) \hat{a} ] \\
&= -\omega \hat{a}
\end{align*}
and
\begin{align*}
[ \hat{H}, \hat{a}^\dagger ] &= \omega \left\lbrace ( \hat{a}^\dagger \hat{a} + \tfrac{1}{2} ) \hat{a}^\dagger - \hat{a}^\dagger ( \hat{a}^\dagger \hat{a} + \tfrac{1}{2} ) \right\rbrace \\
&= \omega ( \hat{a}^\dagger \hat{a} \hat{a}^\dagger - \hat{a}^\dagger \hat{a}^\dagger \hat{a} ) \\
&= \omega [ \hat{a}^\dagger \hat{a} \hat{a}^\dagger - \hat{a}^\dagger ( \hat{a} \hat{a}^\dagger - 1 ) ] \\
&= \omega \hat{a}^\dagger
\end{align*}

% Page 5

\noindent Label the states by
\begin{equation*}
\ket{n} \qquad E_n = ( n + \tfrac{1}{2} ) \omega \qquad \hat{H} \ket{n} = ( n + \tfrac{1}{2} ) \omega \ket{n}
\end{equation*}
and look at
\begin{align*}
\hat{H} \hat{a} \ket{n} &= \omega ( \hat{a}^\dagger \hat{a} + \tfrac{1}{2} ) \hat{a} \ket{n} \\
&= \omega ( \hat{a}^\dagger \hat{a} \hat{a} + \tfrac{1}{2} \hat{a} ) \ket{n} \nonumber\\
&= \omega \left[ ( \hat{a} \hat{a}^\dagger - 1 ) \hat{a} + \tfrac{1}{2} \hat{a} \right] \ket{n} \\
&= \omega \hat{a} ( \hat{a}^\dagger \hat{a} + \tfrac{1}{2} - 1 ) \ket{n} \\
&= \hat{a} ( \hat{H} - \omega ) \ket{n} \\
&= \hat{a} \left[ ( \hat{n} + \tfrac{1}{2} ) \omega - \omega \right] \ket{n} \\
&= ( n - \tfrac{1}{2} ) \omega \hat{a} \ket{n}
\end{align*}
\begin{equation*}
\Rightarrow \quad \hat{a} \ket{n} \propto \ket{n-1}
\end{equation*}

\noindent What is the constant of proportionality?

% Page 6

\noindent From
\begin{equation*}
\hat{H} \ket{n} = ( \hat{a}^\dagger \hat{a} + \tfrac{1}{2} ) \omega \ket{n} = ( n + \tfrac{1}{2} ) \omega \ket{n}
\end{equation*}
we see that the number operator is
\begin{equation*}
\hat{N} = \hat{a}^\dagger \hat{a}
\end{equation*}

\noindent Then
\begin{equation*}
\braket{ n | \hat{a}^\dagger \hat{a} | n } = n
\end{equation*}

\noindent But
\begin{equation*}
\hat{a} \ket{n} = c \ket{n-1}
\end{equation*}
and
\begin{equation*}
\bra{n} \hat{a}^\dagger = c^* \bra{n-1}
\end{equation*}

\noindent Then
\begin{equation*}
n = \braket{ n | \hat{a}^\dagger \hat{a} | n } = {|c|}^2 \braket{ n-1 | n-1 } = {|c|}^2
\end{equation*}
and
\begin{equation*}
c = \sqrt{n}
\end{equation*}

% Page 7

\noindent That is
\begin{equation*}
\hat{a} \ket{n} = \sqrt{n} \ket{n-1}
\end{equation*}

\noindent Now look at
\begin{align*}
\hat{H} \hat{a}^\dagger \ket{n} &= \omega ( \hat{a}^\dagger \hat{a} + \tfrac{1}{2} ) \hat{a}^\dagger \ket{n} \\
&= \omega ( \hat{a}^\dagger \hat{a} + \tfrac{1}{2} ) \hat{a}^\dagger \ket{n} \\
&= \omega ( \hat{a}^\dagger \hat{a} \hat{a}^\dagger + \tfrac{1}{2} \hat{a}^\dagger ) \ket{n} \qquad \qquad \hat{a} \hat{a}^\dagger - \hat{a}^\dagger \hat{a} = 1 \\
&= \omega \hat{a}^\dagger [ ( 1 + \hat{a}^\dagger \hat{a} ) + \tfrac{1}{2} ] \ket{n} \\
&= \hat{a}^\dagger ( \hat{H} + \omega ) \ket{n} \\
&= \hat{a}^\dagger ( n + \tfrac{1}{2} + 1 ) \omega \ket{n} \\
&= ( n + \tfrac{1}{2} + 1 ) \omega \hat{a}^\dagger \ket{n}
\end{align*}
\begin{equation*}
\Rightarrow \quad \hat{a}^\dagger \ket{n} \propto \ket{n+1}
\end{equation*}

% Page 8

\noindent But
\begin{equation*}
\braket{ n+1 | \hat{a}^\dagger \hat{a} | n+1 } = n + 1
\end{equation*}
and
\begin{equation*}
n + 1 = \braket{ n+1 | \hat{a}^\dagger \hat{a} | n+1 } = {|c|}^2 \braket{ n+1 | n+1 } = {|c|}^2
\end{equation*}

\noindent Then
\begin{equation*}
    \hat{a}^\dagger \ket{n} = \sqrt{n+1} \ket{n+1}
\end{equation*}

\noindent One of the modt important aspects of the spectrum of states of the quantized harmonic oscillator is that its energy levels are separated by a uniform amount: $\hbar \omega$. The state $\ket{n}$ has energy $n \hbar \omega$ and the state $\ket{n+1}$ has energy $(n+1) \hbar \omega$.

\noindent With
\begin{equation*}
    \hat{H} \ket{n} = \left( n + \tfrac{1}{2} \right) \hbar \omega \ket{n}
\end{equation*}
we have the vacuum state $\ket{0}$ with energy $\tfrac{1}{2} \hbar \omega$. Then we can think of the state $\ket{n}$ as a state of $n$ quanta each of energy $\hbar \omega$ -- i.e., we can think of it as a \underline{multiparticle} state.

\noindent With this interpretation, the operator $\hat{a}^\dagger$ is a \underline{creation operator} that creates a quantum of energy $\hbar \omega$, whereas $\hat{a}$ is a \underline{annihilation operator} that annihilates a quantum of energy $\hbar \omega$.

% Page 9

\noindent Now we can generate all of the states beginning with the vacuum state defined by
\begin{equation*}
    \hat{a} \ket{0} = 0
\end{equation*}

\noindent Then
\begin{equation*}
    \ket{n} = \dfrac{1}{\sqrt{n!}} {\left( \hat{a}^\dagger \right)}^n \ket{0}
\end{equation*}
$\dfrac{1}{\sqrt{n!}}$ cancels out the $\sqrt{n!}$ that appears when the $\hat{a}^\dagger$ act $n$ times.

\noindent For example,
\begin{align*}
    \ket{2} &= \dfrac{1}{\sqrt{2}} \hat{a}^\dagger \left( \hat{a}^\dagger \ket{0} \right) \\
    &= \dfrac{1}{\sqrt{2}} \hat{a}^\dagger \left( \sqrt{1} \ket{1} \right) \\
    &= \dfrac{1}{\sqrt{2}} \sqrt{1+1} \ket{2} \\
    &= \ket{2}
\end{align*}

\end{document}