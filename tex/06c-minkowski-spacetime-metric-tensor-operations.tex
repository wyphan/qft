\documentclass{article}

\usepackage{amsmath,amssymb}
\usepackage{mathrsfs}
\usepackage{tensor}
\usepackage{fullpage}

\usepackage{textpos}
\usepackage{eso-pic}
\usepackage{tikz}
\usetikzlibrary{tikzmark}

\title{Minkowski Spacetime, Its Metric, and Tensor Operations}
\author{Anthony Mezzacappa}
\date{September 10, 2020}
% Typeset by Wileam Phan

\begin{document}
\setlength{\parskip}{1em}
\maketitle

\begin{align*}
    \tikzmarknode{ds2}{\mathrm{d} s^2} &= -c^2 ~ \mathrm{d}t^2 + \mathrm{d}x^2 + \mathrm{d}y^2 + \mathrm{d}z^2 \qquad \text{(assumes Cartesian coordinates)} \\
    \tikzmarknode{dt2}{\,} &= -c^2 ~ \mathrm{d}\tau^2 \\
    c^2 ~ \mathrm{d}\tau^2 &= c^2 ~ \mathrm{d}t^2 - \mathrm{d}x^2 - \mathrm{d}y^2 - \mathrm{d}z^2 \underset{c=1}{\longrightarrow} \mathrm{d}t^2 - \mathrm{d}x^2 - \mathrm{d}y^2 - \mathrm{d}z^2 \\
    \mathrm{d}\tau^2 &= - \tikzmarknode{eta}{\tensor{\eta}{_\mu_\nu}} ~ \mathrm{d} \tensor{x}{^\mu} ~ \mathrm{d} \tensor{x}{^\nu} \equiv \tensor{\bar{\eta}}{_\mu_\nu} ~ \mathrm{d} \tensor{x}{^\mu} ~ \mathrm{d} \tensor{x}{^\nu}
\end{align*}
{%
\begin{textblock*}{1in}(0.0in,-1.15in)%
\begin{minipage}[h!]{1in}
    \tikzmarknode{ds2label}{proper length}
\end{minipage}%
\end{textblock*}%
}%
{%
\begin{textblock*}{1in}(0.0in,-0.9in)%
\begin{minipage}[h!]{1in}
    \tikzmarknode{dt2label}{proper time}
\end{minipage}%
\end{textblock*}%
}
{%
\begin{textblock*}{3in}(1.55in,-0.2in)%
\begin{minipage}[h!]{3in}
    \tikzmarknode{metric}{M}inkowski metric = %
    $\begin{pmatrix}
    -1 & 0 & 0 & 0 \\
    0  & 1 & 0 & 0 \\
    0  & 0 & 1 & 0 \\
    0  & 0 & 0 & 1
    \end{pmatrix}$
\end{minipage}%
\end{textblock*}%
}
\begin{tikzpicture}[overlay, remember picture]
    \draw[overlay,->] (ds2label.east) -- (ds2.west);
    \draw[overlay,->] (dt2label.east) -- (dt2.west);
    \draw[overlay,->] (metric.north) -- (eta.south);
\end{tikzpicture}

\begin{align*}
    \tensor{\eta}{_0_0} &= -1 \\
    \tensor{\eta}{_i_i} &= 1 \\
    \tensor{\eta}{_\mu_\nu} &= 0 \qquad \mu \neq \nu \\
\end{align*}

\vspace{-48pt} \begin{align*}
    \mathrm{d}\tensor{x}{^0} &= \mathrm{d}t \\
    \mathrm{d}\tensor{x}{^1} &= \mathrm{d}x \\
    \mathrm{d}\tensor{x}{^2} &= \mathrm{d}y \\
    \mathrm{d}\tensor{x}{^3} &= \mathrm{d}z
\end{align*}

\noindent All tensors are built from contravariant vectors (vectors) and covariant vectors (covectors or 1-forms).

\section*{Contravariant Vector}
$\vec{V} : T_P M \rightarrow \mathbb{R}$
\begin{equation*}
    \vec{V} = \tikzmarknode{cvvmu}{\tensor{V}{^\mu}} \tensor{\hat{\mathbf{e}}}{_\mu} \qquad \tikzmarknode{mulabel}{\text{index}} \text{ \underline{ on top} (\underline{contra}variant)} 
\end{equation*}
\begin{tikzpicture}[overlay, remember picture]
    \draw[overlay,->] (mulabel.north) -- +(0,0.2) -- +(-1.5,0.2) -- (cvvmu.north east);
\end{tikzpicture}

% Page 2

\noindent N.B. The Einstein index convention is at play here:
\begin{equation*}
    \vec{V} = \underbrace{ \tensor{V}{^\mu} \tensor{\hat{\mathbf{e}}}{_\mu} } = \tensor{V}{^0} \tensor{\hat{\mathbf{e}}}{_0} + \tensor{V}{^1} \tensor{\hat{\mathbf{e}}}{_1} + \tensor{V}{^2} \tensor{\hat{\mathbf{e}}}{_2} + \tensor{V}{^3} \tensor{\hat{\mathbf{e}}}{_3}
\end{equation*}
\hspace{2.2in} sum over repeated indices

\section*{Covariant Vector}
$\tilde{V} : T_P M^* \rightarrow \mathbb{R}$
\begin{equation*}
    \tilde{V} = \tikzmarknode{1fmu}{\tensor{V}{_\mu}} \tikzmarknode{dxmu}{\tensor{\tilde{\mathrm{dx}}}{^\mu}}
\end{equation*}
{%
\begin{textblock*}{1in}(2.4in,-0.15in)%
\begin{minipage}[h!]{1in}
    \tikzmarknode{mulabel2}{index down}
\end{minipage}%
\end{textblock*}%
}%
{%
\begin{textblock*}{3in}(3.5in,-0.2in)%
\begin{minipage}[h!]{3in}
    \tikzmarknode{dxmulabel}{1}-form \underline{dual} to $\tensor{\hat{\mathbf{e}}}{_\mu}$ -- i.e., $\tensor{\tilde{\mathrm{dx}}}{^\mu} ( \tensor{\hat{\mathbf{e}}}{_\nu} ) = \tensor{\delta}{^\mu_\nu} $
\end{minipage}%
\end{textblock*}%
}
\begin{tikzpicture}[overlay, remember picture]
    \draw[overlay,->] (mulabel2.east) -- (1fmu.south);
    \draw[overlay,->] (dxmulabel.east) -- (dxmu.south);
\end{tikzpicture}

\vspace{-24pt} \begin{equation*}
    \tilde{V} (\vec{V}) = \tilde{V} ( \tensor{V}{^\alpha} \tensor{\hat{\mathbf{e}}}{_\alpha} ) = \tensor{V}{_\beta} \tensor{V}{^\alpha} \tensor{\tilde{\mathrm{dx}}}{^\beta} ( \tensor{\hat{\mathbf{e}}}{_\alpha} )
\end{equation*}

\noindent General Tensors are created via the \underline{Tensor Product}. Let's look at the important case of the \underline{Metric Tensor}

\begin{equation*}
\mathbf{\eta} \equiv \tensor{\eta}{_\mu_\nu} \tensor{\tilde{\mathbf{dx}}}{^\mu} \,\tikzmarknode{otimes}{\otimes}\, \tensor{\tilde{\mathbf{dx}}}{^\nu}
\end{equation*}
{%
\begin{textblock*}{1in}(3in,0.0in)%
\begin{minipage}[h!]{3in}
    \tikzmarknode{tprod}{tensor product}
\end{minipage}%
\end{textblock*}%
}
\begin{tikzpicture}[overlay, remember picture]
    \draw[overlay,->] (tprod.north) -- (otimes.south);
\end{tikzpicture}

Then
\begin{align*}
	\vec{u} \tikzmarknode{inprod}{\cdot} \vec{v} \equiv \eta ( \tikzmarknode{uvec}{\vec{u}}, \tikzmarknode{vvec}{\vec{v}} ) &= \tensor{\eta}{_\mu_\nu} \tensor{\tilde{\mathbf{dx}}}{^\mu} \otimes \tensor{\tilde{\mathbf{dx}}}{^\nu} ( \vec{u}, \vec{v} ) \\
	&= \tensor{\eta}{_\mu_\nu} \tensor{u}{^\alpha} \tensor{v}{^\beta} \tikzmarknode{dxmu}{\underbrace{\tensor{\tilde{\mathbf{dx}}}{^\mu} ( \tensor{\hat{\mathbf{e}}}{_\alpha} ) }} \tikzmarknode{dxnu}{\underbrace{\tensor{\tilde{\mathbf{dx}}}{^\nu} ( \tensor{\hat{\mathbf{e}}}{_\beta} ) }} \\
	&= \tensor{\eta}{_\mu_\nu} \tensor{u}{^\alpha} \tensor{v}{^\beta} \tikzmarknode{dma}{\tensor{\delta}{^\mu_\alpha}} \tikzmarknode{dnb}{\tensor{\delta}{^\nu_\beta}} \\
	&= \tensor{\eta}{_\mu_\nu} \tensor{u}{^\alpha} \tensor{v}{^\beta}
\end{align*}
\begin{tikzpicture}[overlay, remember picture]
    \draw[overlay,->] (dxmu.south) -- (dma.north);
    \draw[overlay,->] (dxnu.south) -- (dnb.east);
\end{tikzpicture}

% Page 3

\noindent Since this is just a scalar (a number), the following must be a 1-form
\begin{align*}
	\tikzmarknode{eta}{\eta} ( \vec{u}, \,\tikzmarknode{empty}{.}\, ) &= \tensor{\eta}{_\mu_\nu} \tensor{\tilde{\mathbf{dx}}}{^\mu} \otimes \tensor{\tilde{\mathbf{dx}}}{^\nu} ( \vec{u}, \,.\, ) \\
	&= \tensor{\eta}{_\mu_\nu} \tensor{u}{^\alpha} \tensor{\tilde{\mathbf{dx}}}{^\mu} \otimes \tensor{\tilde{\mathbf{dx}}}{^\nu} ( \tensor{\hat{\mathbf{e}}}{_\alpha}, \,.\, ) \\
	&= \tensor{\eta}{_\mu_\nu} \tensor{u}{^\alpha} \tensor{\delta}{^\mu_\alpha} \tensor{\tilde{\mathbf{dx}}}{^\nu} ( \,.\, ) \\
	&= \tensor{\eta}{_\mu_\nu} \tensor{u}{^\mu} \tensor{\tilde{\mathbf{dx}}}{^\nu} ( \,.\, ) \\
	&= \tensor{u}{_\nu} \tensor{\tilde{\mathbf{dx}}}{^\nu} ( \,.\, )
\end{align*}

\noindent That is, the metric raises and lowers indices (the inverse, $\tensor{\eta}{^\mu^\nu}$, would raise indices)
\begin{align*}
	\tensor{u}{_\nu} &= \tensor{\eta}{^\mu^\nu} \tensor{u}{^\mu} \\
	\tensor{u}{_\mu} &= \tensor{\eta}{^\mu^\nu} \tensor{u}{_\nu}	
\end{align*}

% Page 4

\noindent So the inner product
\begin{align*}
	\vec{u} \cdot \vec{v} &= \tensor{\eta}{_\alpha_\beta} \tensor{u}{^\alpha} \tensor{v}{^\beta} \\
	&= \tensor{u}{_\beta} \tensor{\tilde{\mathbf{dx}}}{^\beta} ( \tensor{v}{^\alpha} \tensor{\hat{\mathbf{e}}}{_\alpha} ) = \tilde{u} ( \vec{v} ) \\
	&= \tensor{u}{_\beta} \tensor{v}{^\beta} \\
	&= \tensor{u}{_\beta} \tensor{v}{^\alpha} \tensor{\tilde{\mathbf{dx}}}{^\beta} ( \tensor{\hat{\mathbf{e}}}{_\alpha} ) \\
	&= \tensor{u}{_\beta} \tensor{v}{^\alpha} \tensor{\delta}{^\beta_\alpha} \\
	&= \tensor{u}{_\beta} \tensor{v}{^\beta}
\end{align*}

\noindent In Euclidean space in Cartesian coordinates, we write
\begin{equation*}
	\vec{u} \cdot \vec{v} \equiv \tensor{u}{^1} \tensor{v}{^1} + \tensor{u}{^2} \tensor{v}{^2} + \tensor{u}{^3} \tensor{v}{^3} \tikzmarknode{euclid}{\,}
\end{equation*}
{%
\begin{textblock*}{1.7in}(4in,-0.2in)%
\begin{minipage}[h!]{1.7in}
    \tikzmarknode{diffgeom}{there's} a lot of differential geometry hidden here
\end{minipage}%
\end{textblock*}%
}
\begin{tikzpicture}[overlay, remember picture]
    \draw[overlay,->] (diffgeom.west) -- (euclid.east);
\end{tikzpicture}
which is really
\begin{equation*}
	\tensor{u}{_1} \tensor{v}{^1} + \tensor{u}{_2} \tensor{v}{^2} + \tensor{u}{_3} \tensor{v}{^3} \tikzmarknode{euclid}{\,}
\end{equation*}

\noindent But, given the metric
\begin{equation*}
	\tensor{\eta}{_i_j} = %
	\begin{pmatrix}
		1 &   & 0 \\
		  & 1 &   \\
		0 &   & 1
	\end{pmatrix}
\end{equation*}
we have
\begin{equation*}
	\tensor{u}{_1} = \tensor{\eta}{_1_i} \tensor{u}{^i} = \tensor{\eta}{_1_1} \tensor{u}{^1} = (+1) \tensor{u}{^1} = \tensor{u}{^1}
\end{equation*}
so it doesn't matter.

% Page 5

\noindent But in Minkowski space, it \underline{does} matter.

\noindent Note that
\begin{equation*}
	\vec{u} \cdot \vec{v} = \eta ( \vec{u}, \vec{v} ) = \tensor{u}{_\alpha} \tensor{v}{^\alpha}
\end{equation*}
is just a number. As a scalar, it is Lorentz invariant. That means that $\tensor{u}{_\alpha}$ and $\tensor{v}{^\alpha}$ must transform under a Lorentz transformation in an inverse manner.
\begin{align*} % tensor package doesn't like \tensor{V}{^\mu'}
	\tensor{V}{^\mu^\prime} &= \dfrac{ \partial \tensor{x}{^\mu^\prime} }{ \partial \tensor{x}{^\mu} } \tensor{V}{^\mu} \\
	\tensor{V}{_\mu_\prime} &= \dfrac{ \partial \tensor{x}{^\mu} }{ \partial \tensor{x}{^\mu^\prime} } \tensor{V}{_\mu} \\
\end{align*}

\noindent Then
\begin{align*}
	\tensor{V}{^\mu^\prime} \tensor{V}{_\mu_\prime} &= \dfrac{ \partial \tensor{x}{^\mu^\prime} }{ \partial \tensor{x}{^\mu} } \dfrac{ \partial \tensor{x}{^\mu} }{ \partial \tensor{x}{^\mu^\prime} } \tensor{V}{^\mu} \tensor{V}{_\mu} \\
	&= \tensor{V}{^\mu} \tensor{V}{_\mu}
\end{align*}

\noindent Some important specific tensors:
\begin{itemize}

	\item The differential (a 1-form)
	\begin{equation*}
	\tilde{\mathrm{d}f} = \dfrac{ \partial f }{ \partial \tensor{x}{^\mu} } ~ \tilde{ \mathrm{d} \tensor{x}{^\mu} } \equiv \tensor{\partial}{_\mu} f ~ \tilde{ \mathrm{d} \tensor{x}{^\mu} } 
	\end{equation*}

% Page 6
	
	\item The gradient (a contravariant vector) is defined by
	\begin{align*}
		\tilde{\mathrm{d}f} ( \tensor{\hat{\mathbf{e}}}{_\nu} ) &= \eta ( \vec{\nabla} f, \tensor{\hat{\mathbf{e}}}{_\nu} ) \\
		&= \eta \left( \tensor{( \vec{\nabla} f )}{^\mu} ~ \tensor{\hat{\mathbf{e}}}{_\mu}, \tensor{\hat{\mathbf{e}}}{_\nu} \right) \\
		&= \tensor{( \vec{\nabla} f )}{^\mu} ~ \eta ( \tensor{\hat{\mathbf{e}}}{_\mu}, \tensor{\hat{\mathbf{e}}}{_\nu} ) \\
		&= \tensor{\eta}{_\mu_\nu} ~ \tensor{( \vec{\nabla} f )}{^\mu}
	\end{align*}
	But
	\begin{equation*}
		\tilde{\mathrm{d}f} ( \tensor{\hat{\mathbf{e}}}{_\nu} ) = \dfrac{ \partial f }{ \partial \tensor{x}{^\nu} }
	\end{equation*}
	Then
	\begin{equation*}
		\dfrac{ \partial f }{ \partial \tensor{x}{^\nu} } = \tensor{\eta}{_\mu_\nu} ~ \tensor{( \vec{\nabla} f )}{^\mu} \tikzmarknode{part1}{\,}
	\end{equation*}
	or
	\begin{equation*}
		\tensor{( \vec{\nabla} f )}{^\mu} = \tensor{\eta}{^\mu^\nu} \dfrac{ \partial f }{ \partial \tensor{x}{^\nu} } \tikzmarknode{part2}{\,}
	\end{equation*}
	{%
	\begin{textblock*}{2in}(4.1in,-0.8in)%
	\begin{minipage}[h!]{2in}
		\begin{align*}
		   \tensor{\eta}{^\alpha^\nu} ~ \tensor{\eta}{_\mu_\nu} ~ \tensor{( \vec{\nabla} f )}{^\mu} &= \tensor{\delta}{^\alpha_\mu} ~ \tensor{( \vec{\nabla} f )}{^\mu} \\
			\tensor{\eta}{^\alpha^\nu} ~ \dfrac{ \partial f }{ \partial \tensor{x}{^\nu} } &= \tensor{( \vec{\nabla} f )}{^\alpha}
		\end{align*}
	\end{minipage}%
	\end{textblock*}%
	}
	\begin{tikzpicture}[overlay, remember picture]
		\draw[overlay,->] (part1.west) arc (90:-90:.6) -- (part2.west);
	\end{tikzpicture}

% Page 7

	\noindent Since
	\begin{equation*}
		\dfrac{ \partial f }{ \partial \tensor{x}{^\mu} } = \tensor{\partial}{_\mu} f
	\end{equation*}
	we must have
	\begin{align*}
		\tensor{( \vec{\nabla} f )}{^\mu} &= \tensor{\eta}{^\mu^\nu} ~ \tensor{\partial}{_\nu} f \\
		&= \tensor{\partial}{^\nu} f
	\end{align*}

	\item Since
	\begin{equation*}
		\tensor{\partial}{_\mu} = \dfrac{\partial}{ \partial \tensor{x}{^\mu} }
	\end{equation*}
	we must have
	\begin{equation*}
		\tensor{\partial}{^\mu} = \dfrac{\partial}{ \partial \tensor{x}{_\mu} }
	\end{equation*}

	\item The $\tensor{x}{_\mu}$ are the components of the 1-form dual to the position vector
	\begin{equation*}
		\vec{x} = \tensor{x}{^\mu} \tensor{\hat{\mathbf{e}}}{_\mu}
	\end{equation*}
	-- i.e.,
	\begin{equation*}
		\tilde{x} = \tensor{x}{_\mu} \tilde{ \mathrm{d} \tensor{x}{^\mu} }
	\end{equation*}

\end{itemize}

\end{document}